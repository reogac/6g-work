\documentclass[a4paper]{article}
\usepackage{hyperref}
\author{TungTQ}
\title{Quotes related to 5G and cloud native}

\begin{document}
from \href{https://www.datocms-assets.com/2885/1654902115-hashicorp_a_cloud_operating_model_for_platform_teams.pdf}{hashicorp's whitepaper}
\begin{quote}
Cloud is now the default choice for organizations delivering new value to their customers.
Successful companies use a cloud operating model — a framework for adopting cloud
services — to maximize agility, reliability, and security and deliver superior business
outcomes.
\end{quote}%
\begin{quote}
Dynamic cloud infrastructure means a shift from host-based identity to application-based identity, with
low or zero trust networks across multiple clouds without a clear network perimeter.
\end{quote}
\begin{quote}
Networking services should be based on service identity and provided centrally, allowing platform
teams to create a centralized service registry for discovery purposes. The common registry provides a
“map” of what services are running, where they are, and their current health. The registry can be queried
programmatically to enable service discovery or drive network automation of API gateways, load
balancers, firewalls, and other critical middleware components. Service mesh approaches can simplify
the network topology, especially in multi-cloud and multi-datacenter environments.
\end{quote}

from \href{https://www.redhat.com/architect/5g-open-hypercore-architecture}{a blog post on hypercore architechture}
\begin{quote}
To expand 5G coverage to a wider consumer base, communication service providers need visibility into the return on investment (ROI) for the capital expenditure (CapEx) investments to acquire, deliver, and build 5G, as well as the ongoing maintenance and operational expenditures (OpEx). Collectively, these place a significant financial burden on service providers.

Telco service providers could benefit from adaptable, on-demand, and pay-as-they-consume infrastructure services—from 5G Core (5GC), to enterprise/consumer network edge, and even towards 5G radio access network (RAN)—that address diverse use cases while minimizing CapEx and OpEx.
\end{quote}

from \href{https://www.redhat.com/rhdc/managed-files/cl-architect-guide-multicloud-infrastructure-ebook-f29609-202108-en.pdf}{Kubernetes defined - redhat book on Kubernetes}
\begin{quote}
Kubernetes defined
Kubernetes is an open source container
orchestration platform that automates many
of the manual processes involved in deploying,
managing, and scaling containerized
applications.

Accepted by many as the de facto control plane
for managing and deploying containers,
Kubernetes can also help you deliver and
manage containerized, traditional, and
cloud-native apps at scale.

While it may seem like containers introduce new
layers of complexity, Kubernetes applies
automation to streamline operations.
\end{quote}
\end{document}